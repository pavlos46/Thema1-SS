\documentclass{article}
\usepackage[greek,english]{babel}
\usepackage[LGR,T1]{fontenc}
\usepackage[utf8]{inputenc}
\usepackage{graphicx}
\usepackage{listings}

\title{Αναφορά για τον Φιλτραρισμό Ηχητικού Σήματος}
\author{Pavlos Saridis AEM 3732}
\date{\today}

\begin{document}
\maketitle

\section*{Εισαγωγή}
Στην παρούσα αναφορά παρουσιάζεται ο φιλτραρισμός ενός ηχητικού σήματος με χρήση πεπερασμένης κρουστικής απόκρισης φίλτρου χαμηλής συχνότητας.

\section*{Μεθοδολογία}

Για την υλοποίηση του φιλτραρίσματος χρησιμοποιήθηκε ο κώδικας σε Python που παρατίθεται παρακάτω. Ακολουθήστε τα σχόλια για κατανόηση των βημάτων.

\begin{lstlisting}[language=Python, caption=Κώδικας Python για τον φιλτραρισμό ηχητικού σήματος]
# Εισαγωγή των απαραίτητων βιβλιοθηκών
import numpy as np
import matplotlib.pyplot as plt
from scipy.io import wavfile

# Αλλαγή του Matplotlib backend σε Agg
import matplotlib
matplotlib.use('Agg')

# Φόρτωση αρχείου WAV
samplerate, data = wavfile.read('filtered_output.wav')

# Σχεδιασμός και εφαρμογή του φίλτρου
def design_lowpass_filter(samplerate, cutoff=0.2, numtaps=300):
    nyquist = 0.5 * samplerate
    cutoff_normalized = cutoff / nyquist
    taps = np.sinc(2 * cutoff_normalized * np.arange(numtaps))
    taps *= np.blackman(numtaps)
    taps /= np.sum(taps)
    return taps

def apply_filter(data, taps):
    filtered_data = np.convolve(data[:, 0], taps, mode='same')
    return filtered_data

# Σχεδιασμός του φίλτρου
taps = design_lowpass_filter(samplerate, cutoff=0.2)

# Εφαρμογή του φίλτρου στο σήμα
filtered_data = apply_filter(data, taps)

# Προαιρετική, σχεδίαση της συχνοτικής απόκρισης
w = np.fft.fftfreq(len(taps))
response = np.fft.fft(taps)
plt.plot(0.5 * samplerate * w, np.abs(response), 'b')
plt.xlabel('Συχνότητα (Hz)')
plt.ylabel('Κέρδος')
plt.title('Συχνοτική Απόκριση του Φίλτρου')
plt.ylim(-0.05, 1.05)
plt.grid(True)
plt.savefig('frequency_response.png')
plt.close()

# Αποθήκευση του φιλτραρισμένου αρχείου
wavfile.write('filtered_output.wav', samplerate, filtered_data.astype(data.dtype))
\end{lstlisting}

\end{document}
